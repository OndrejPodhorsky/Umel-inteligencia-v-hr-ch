\documentclass{article}
\usepackage[utf8]{inputenc}

\title{Umelá inteligencia v hrách\thanks{Semestrálny projekt v predmete Metódy inžinierskej práce, ak. rok 2021/22, vedenie: Igor Stupavský}} 

\author{Ondrej Podhorsky\\[2pt]
	{\small Slovenská technická univerzita v Bratislave}\\
	{\small Fakulta informatiky a informačných technológií}\\
	{\small \texttt{xpodhorsky@stuba.sk}}
	}

\date{\small 6. november 2022}

\begin{document}

\maketitle

\begin{abstract}
\ldots
\end{abstract}

Cieľom tohto článku je priblížiť použitie a možnosti použitia umelej inteligencie v video hernom priemysle, ktorý sa v súčasnosti dostáva do popredia v zábavnom priemysle. V roku 2022 by mal podľa očakávaní mať výnos 197 biliónov dolárov, čo by ho radilo napríklad pred hudobný alebo filmový priemysel. Dlhodobo sa snažili najpoprednejšie štúdia vyvíjať hry s dôrazom  na grafickú realistickosť. No v podstate tento ciel už bol dosiahnutý a preto sa snažia implementovať iné technológie do svojich hier, ktorými by sa dal obohatiť video herný zážitok. Umelá inteligencia neprináša prínos len z tohto hľadiska ale aj napríklad pri samotnom vývine či testovaní hier.

\section{Introduction}

\end{document}
