\documentclass{article}
\usepackage[utf8]{inputenc}
\usepackage{graphics}
\usepackage{graphicx}

\author{Ondrej Podhorsky\\[2pt]
	{\small Slovenská technická univerzita v Bratislave}\\
	{\small Fakulta informatiky a informačných technológií}\\
	{\small \texttt{xpodhorsky@stuba.sk}}
	}

\date{\small 6. november 2022}

\begin{document}

\maketitle

\section{Abstrakt}

Obsahom tohto článku je čitateľovi priblížiť aplikáciu umelej inteligencie vo video hernom priemysle, dejiny a minulosť uplatnenia. Umelá inteligencia ako má tento článok ukázať, má za ciel riešenie komplexných problémov pomocou stroja v hrách, čiže napríklad v oblastiach samotného vývoja, fungovania a hrateľnosti . Zároveň takáto aplikácia takýchto algoritmov má v súčasnosti určité limitácie na, ktoré bude snaha poukázať a prípadne nájsť použitelné a vhodné možnosti riešenia.

\clearpage

\section{Úvod}

Cieľom tohto článku je priblížiť použitie a možnosti použitia umelej inteligencie v video hernom priemysle, ktorý sa v súčasnosti dostáva do popredia v zábavnom priemysle. V roku 2022 by mal podľa očakávaní mať výnos 225 biliónov dolárov \cite{TeodoraDobrilovat}, čo by ho radilo napríklad pred hudobný alebo filmový priemysel. Dlhodobo sa snažili najpoprednejšie štúdia vyvíjať hry s dôrazom  na grafickú realistickosť. No v podstate tento ciel už bol dosiahnutý a preto sa snažia implementovať iné technológie do svojich hier, ktorými by sa dal obohatiť video herný zážitok, zjednodušiť ich vývin či testovanie hier. Prvá časť bude obsahovať informácie o disciplíne, ktorú použitie umelej iteligencie definovalo. V ďalších častiach sa bude písať o aspektoch hry, ako je napríklad tvorba a fungovanie nehráčských postáv a jej limitáciám, vylepšením tvorby herného zážitku implentovaním umelej inteligencie do herných mechaník a generovaním obashu pre zjednodušenie práce herných vývojárov.

\begin{figure}{1. Aplikácia umelej inteligencie v hrách\cite{inproceedings}}
\centering
\includegraphics[width=103mm]{Game-AI-Applications.png}
\end{figure}

\clearpage

\section{Hry ako testovacie prostredie}

Hry boli odzačiatku vzniku prvých algoritmov pre vedcov ihrysko, kde mohli testovať najnovšie technológie. V roku 1951 Christopher Strachey  napísal program pre dámu a Dietrich Prinz pre šach \cite{Copeland}. Klasické doskové hry boli ideálne pre testovanie s pevnými pravidlami no zároveň komplexné. Cielom bola snaha poraziť človeka v týchto hrách. Veľkým milníkom sa stal šachový zápas medzi veľmajstrom Kasparovom a IBM Deep Blue v roku 1997, ktorý Deep Blue na koniec vyhral 4-2. Alebo nedávne výťaztvo Google Deeps mind Alpha go v hre go nad svetovým šampión Lee Sedolom. Go je najkomplexnejšia dosková z klasických abstraktných hier a preto sa v súčastnosti prechádza na kartové hry či online stratégie odohrávajúce sa v reálnom čase.

\subsection{Moba, kartové hry}

\section{Nehráčske postavy}

Nehráčske postavy sú neodmysliteľnou časťou hier od nepamäti. Sú to prevažne charaktery, ktoré ovláda počítač po prípade hráč. No v súčasnosti je to časť hry, ktorá sa často prehliada a priorita sa hlavne prikladá na grafickú stránku hier. Umelá inteligencia má potenciál tento problém vyriešiť. Momentálne sa prevažne používajú primitívne postupy vytvorené ešte v 90tich rokoch, ktoré sú napríklad pathfinding,finite state machines alebo komplexnejšie ako je napríklad Monte Carlo Tree Search\cite{nickstatt}. Vývojári v posledných dekádach len zväčšujú merítko v akom sú tieto koncepty využívané.

\subsection{Limitácie}

Hlavnými obmedzeniami sú nedostatočné výpočtové prostriedky dostupné pre bežného používateľa. Tak tiež nepredvídateľnosť takýchto charakterov by vytvárala chaos a práve naopak by ešte viac znížila pocit prirodzeného správania a podkopala tak video herný zážitok\cite{phdthesis}.

\section{Tvorba herného zážitku}

\section{Generovanie herného obsahu}
\section{Záver}

\bibliography{literatura}
\bibliographystyle{plain}
\end{document}
